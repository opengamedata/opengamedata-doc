%!LW recipe=beamer - local build dir
\documentclass{beamer}
% Based on template by Till Tantau <tantau@users.sourceforge.net>.

% \usepackage[english]{babel}
\usepackage[latin1]{inputenc}
% or whatever
\usepackage[T1]{fontenc}
% Or whatever. Note that the encoding and the font should match. If T1
% does not look nice, try deleting the line with the fontenc.
\usepackage{graphicx}

\mode<presentation>
{
  \def\ogd@assetdir{../../../assets/extern}
  \usetheme{frankfurt}
  % \usepackage[imagedir=\ogd@assetdir/images]{\ogd@assetdir/style/latex/beamercolorthemeogd}
  \usepackage[imagedir=\ogd@assetdir/images]{../../theme/beamercolorthemeogd}

  \setbeamercovered{transparent}
  % or whatever (possibly just delete it)
}

\title[Intro to Events] % (optional, use only with long paper titles)
{Lecture 2: Introduction to Event Logging in OpenGameData}

\author[Swanson] % (optional, use only with lots of authors)
{Luke Swanson}
\institute[University of Wisconsin-Madison] % (optional, but mostly needed)
{
  Field Day Lab\\
  University of Wisconsin-Madison
}

\date[OGD Docs] % (optional)
{OpenGameData Documentation \\ Unit 1: Event Logging}

\subject{OGD Events}
% This is only inserted into the PDF information catalog. Can be left out. 

% Delete this, if you do not want the table of contents to pop up at
% the beginning of each subsection:
\AtBeginSubsection[]
{
  \begin{frame}<beamer>{Outline}
    \tableofcontents[currentsection,currentsubsection]
  \end{frame}
}

% If you wish to uncover everything in a step-wise fashion, uncomment
% the following command: 
\beamerdefaultoverlayspecification{<+->}


\begin{document}

\begin{frame}
  \titlepage
\end{frame}

\begin{frame}{Overview}
  \tableofcontents
  % You might wish to add the option [pausesections]
\end{frame}

\section[Event Schema]{The OpenGameData Event Schema}

\begin{frame}{What is the OGD Event Schema?}
  \begin{itemize}
    \item \alert{OGD Event Schema :} A standardized set of elements for game telemetry events.
    \item The schema is comprised of elements in the following categories
    \begin{itemize}
      \item IDs
      \item Versioning
      \item Timing
      \item Data (the good stuff!)
    \end{itemize}
  \end{itemize}
\end{frame}

\begin{frame}{Why the OGD Event Schema?}
  \begin{itemize}
    \item Events vary greatly across games
    \item Using a standard allows consistent event interface across games
  \end{itemize}
\end{frame}

\subsection[IDs]{Identifiers}

\begin{frame}{What Identifiers Do We Use?}
  \begin{itemize}
    \item \alert{App ID :} An identifier for the specific game an event came from, typically a single word from the game title
    \begin{itemize}
      \item e.g. AQUALAB, BLOOM, WEATHER\_STATION
    \end{itemize}
    \item \alert{User ID :} A unique identifier (within given game) for an individual user, who may have one or more sessions of play
    \item \alert{Session ID :} A unique identifier for an individual gameplay session, from the time the game is opened to the time it is closed
  \end{itemize}
\end{frame}

\subsection[versions]{Versioning}

\begin{frame}{What Versioning Do We Use?}
  \begin{itemize}
    \item \alert{Log Version :} The version of the logging code (not the game experience).
    \begin{itemize}
      \item Increment whenever the given game's logging schema changes
    \end{itemize}
    \item \alert{App Version :} The version of the game experience
    \begin{itemize}
      \item Use the game's internal versioning
    \end{itemize}
    \item \alert{App Branch :} An optional field to track parallel game versions, e.g. in A/B tests
    \begin{itemize}
      \item Matches well to branch usage in a version control system
    \end{itemize}
  \end{itemize}
\end{frame}

\subsection[timing]{Timing}

\begin{frame}{How is time tracked?}
  \begin{itemize}
    \item \alert{Timestamp}
  \end{itemize}
\end{frame}

\subsection[data]{General Event Data}

\begin{frame}{What per-Game or per-Event data Do We Use?}
  \begin{itemize}
    \item \alert{Event Name} : The name of the specific event type
    \begin{itemize}
      \item e.g. game\_start, level\_complete, click\_place\_object
    \end{itemize}
    \item \alert{Event Source :} Simple indicator for whether event came from the \textbf{GAME}, or was \textbf{GENERATED} by an external \textbf{event detector}.
  \end{itemize}
\end{frame}

\begin{frame}{What per-Game or per-Event data Do We Use?}
  JSON-formatted Data
  \begin{itemize}
    \item All use a JSON-formatted string to allow for differences between systems, games, and events
    \item \alert{User Data}
    \item \alert{Game State}
    \item \alert{Event Data}
  \end{itemize}
\end{frame}

\begin{frame}{User Data}
  Use a common set of keys for all games with a common set of users
  \begin{itemize}
    \item Examples:
    \begin{itemize}
      \item Username
      \item List of achievements
      \item All-time score in current game
    \end{itemize}
  \end{itemize}
\end{frame}

\begin{frame}{Game State}
  \begin{itemize}
    \item Use a common set of keys for all events within a given game
    \item Should be the state at the instant before the event occurred
    \begin{itemize}
      \item For example: a "place object" event would include the score prior to object placement in the Game State, and the score after placement in the Event Data
    \end{itemize}
    \item Example Keys:
    \begin{itemize}
      \item Current level number
      \item Overall score
      \item Score in current level
    \end{itemize}
  \end{itemize}
\end{frame}

\begin{frame}{Event Data}
  \begin{itemize}
    \item Custom set of keys for each event type
  \end{itemize}
\end{frame}

\section[Examples]{Example Games}

\begin{frame}{Simple Example Game}
  \begin{itemize}
  \item Something about an example of a really simple game
  \item So we can imagine doing a thing with it
  \end{itemize}
\end{frame}

\section*{Summary}

\begin{frame}{Summary}
  % Keep the summary *very short*.
  \begin{itemize}
  \item
    The \alert{first main message} of your talk in one or two lines.
  \item
    The \alert{second main message} of your talk in one or two lines.
  \item
    Perhaps a \alert{third message}, but not more than that.
  \end{itemize}
\end{frame}


\end{document}


