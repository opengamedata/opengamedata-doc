\documentclass{beamer}
% Based on template by Till Tantau <tantau@users.sourceforge.net>.

% \usepackage[english]{babel}
\usepackage[latin1]{inputenc}
% or whatever
\usepackage[T1]{fontenc}
% Or whatever. Note that the encoding and the font should match. If T1
% does not look nice, try deleting the line with the fontenc.

\mode<presentation>
{
  \usetheme{frankfurt}
  % or ...

  \setbeamercovered{transparent}
  % or whatever (possibly just delete it)
}

\pgfdeclareimage[height=0.5cm]{ogd-logo}{../../../assets/OpenGameData-logo.png}
\logo{\pgfuseimage{ogd-logo}}

\title[Feature Implementation] % (optional, use only with long paper titles)
{Lecture 8: Feature Implementation}
% \subtitle
% {\texttt{opengamedata-unity} and \texttt{opengamedata-js-log}} % (optional)

\author[Swanson] % (optional, use only with lots of authors)
{Luke Swanson}
% - Use the \inst{?} command only if the authors have different
%   affiliation.

\institute[University of Wisconsin-Madison] % (optional, but mostly needed)
{
  Field Day Lab\\
  University of Wisconsin-Madison
}
% - Use the \inst command only if there are several affiliations.
% - Keep it simple, no one is interested in your street address.

\date[OGD Docs] % (optional)
{OpenGameData Documentation \\ Unit 2: Feature Engineering}

\subject{Talks}
% This is only inserted into the PDF information catalog. Can be left out. 

% Delete this, if you do not want the table of contents to pop up at
% the beginning of each subsection:
\AtBeginSubsection[]
{
  \begin{frame}<beamer>{Outline}
    \tableofcontents[currentsection,currentsubsection]
  \end{frame}
}

% If you wish to uncover everything in a step-wise fashion, uncomment
% the following command: 
% \beamerdefaultoverlayspecification{<+->}


\begin{document}

\begin{frame}
  \titlepage
\end{frame}

\begin{frame}{Overview}
  \tableofcontents
  % You might wish to add the option [pausesections]
\end{frame}


\section[Worked Example]{Worked Example Feature Implementation}

\begin{frame}{A Simple Counter Feature}
  \begin{itemize}
  \item
    // Describe which event type we want to count, what the feature is going to be
    \pause
  \end{itemize}
\end{frame}

\section[\textunderscore\textunderscore{}init\textunderscore\textunderscore{} File]{Adding Feature to Game Module \texttt{\textunderscore\textunderscore{}init\textunderscore\textunderscore{}} File}

\section[Feature Loader]{Registering Feature with Loader}

\section[Feature Config]{Adding Feature to Game Config}

\section*{Summary}

\begin{frame}{Summary}
  % Keep the summary *very short*.
  \begin{itemize}
  \item
    The \alert{first main message} of your talk in one or two lines.
  \item
    The \alert{second main message} of your talk in one or two lines.
  \item
    Perhaps a \alert{third message}, but not more than that.
  \end{itemize}
\end{frame}


\end{document}


